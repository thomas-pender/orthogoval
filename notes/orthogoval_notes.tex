\documentclass[a4paper,11pt]{article}

\usepackage[utf8]{inputenc}
\usepackage[english]{babel}
\usepackage{amssymb, amsmath, amsthm, mathrsfs}
\usepackage[left=1.0in,right=1.0in,top=1.0in,bottom=1.0in]{geometry}
\usepackage[T1]{fontenc}
\usepackage{array}
\usepackage{longtable}
\usepackage{multirow}
\usepackage{calc}
\usepackage[inline,shortlabels]{enumitem}
\usepackage{changepage}
\usepackage{booktabs}
\usepackage{capt-of}
\usepackage{subcaption}
\usepackage[leftcaption]{sidecap}
\usepackage[numbers]{natbib}
\usepackage{times}
\usepackage{titlesec}
\usepackage{xcolor}
\usepackage{lineno}
\usepackage{xpatch}
\xpatchcmd\swappedhead{~}{.~}{}{}
\allowdisplaybreaks

\newtheoremstyle{mythm}
{}                % Space above
{}                % Space below
{}        % Theorem body font % (default is "\upshape")
{1.5em}                % Indent amount
{\scshape}       % Theorem head font % (default is \mdseries)
{.}               % Punctuation after theorem head % default: no punctuation
{0.5em}               % Space after theorem head
{}                % Theorem head spec
\theoremstyle{mythm}
\newtheorem*{theorem*}{Theorem}
\newtheorem{theorem}{Theorem}
\newtheorem{fact}[theorem]{Fact}
\newtheorem{proposition}[theorem]{Proposition}
\newtheorem{lemma}[theorem]{Lemma}
\newtheorem{corollary}[theorem]{Corollary}
\newtheorem{question}[theorem]{Question}
\newtheorem{example}[theorem]{Example}
\newtheorem{definition}[theorem]{Definition}
\newtheorem{conjecture}[theorem]{Conjecture}
\newtheorem{result}[theorem]{Result}
\newtheorem{remark}[theorem]{Remark}
\newtheorem*{remark*}{Remark}
\newtheorem{observation}[theorem]{Observation}

\makeatletter
\renewenvironment{proof}[1][\proofname]{\par
  \pushQED{\qed}%
  \normalfont \topsep6\p@\@plus6\p@\relax
  \trivlist
\item\relax
  {\hspace{1.5em}\itshape
    #1\@addpunct{.}}\hspace\labelsep\ignorespaces
}{%
  \popQED\endtrivlist\@endpefalse
}
\makeatother

\def\Box{\hskip1ex\vbox{\hrule height0.6pt\hbox{%
      \vrule height1.3ex width0.6pt\hskip0.8ex
      \vrule width0.6pt}\hrule height0.6pt
  }}
\renewcommand{\qed}{\Box}

\newcommand{\red}[1]{\textcolor{red}{#1}}
\newcommand{\blue}[1]{\textcolor{blue}{#1}}
\newcommand{\purple}[1]{\textcolor{magenta}{#1}}
\newcommand{\ccite}{\rred{****CITE****}}
\newcommand{\ccirc}{\text{circ}}
\newcommand{\ddet}{\text{det}}
\renewcommand{\pmod}[1]{\text{ (mod $#1$)}}
\newcommand{\mmod}[2]{#1\text{ mod }#2}
\newcommand{\abs}[1]{\left\vert #1 \right\vert}
\newcommand{\C}{\mathbf{C}}
\newcommand{\Z}{\mathbf{Z}}
\newcommand{\LL}{\mathscr{G}}
\newcommand{\z}{\mathbin{\ooalign{$\hidewidth i \hidewidth$\cr$\phantom{+}$}}}
\newcommand{\y}{\mathbin{\ooalign{$\hidewidth j \hidewidth$\cr$\phantom{+}$}}}
\newcommand{\gencite}[1]{\citeauthor{#1}'s~\citep{#1}}
\newcommand{\gf}{\text{GF}}
\newcommand{\gr}{\text{GR}}
\newcommand{\V}{\text{V}}
\newcommand{\re}{\mathfrak{Re}}
\newcommand{\imag}{\mathfrak{Im}}
\newcommand{\smat}[1]{\left(\begin{smallmatrix}#1\end{smallmatrix}\right)}
\newcommand{\bb}{\backslash}

\newcolumntype{R}{>{\scriptsize}r}
\newcolumntype{L}{>{\scriptsize}l}
\newcolumntype{C}{>{\scriptsize}c}

\renewcommand{\citenumfont}[1]{\textbf{#1}}
\renewcommand{\bibnumfmt}[1]{\textbf{#1.}}

\titleformat{\section}{\normalfont\Large\bfseries\centering}{\thesection.}{0.5em}{}
\titleformat{\subsection}{\normalfont\bfseries}{\thesubsection.}{0.5em}{}

\newenvironment{myabstract}{\vspace{1em}\begin{adjustwidth}{3em}{3em}\begin{small}\textbf{Abstract.}}{\end{small}\end{adjustwidth}\vspace{1em}}
\newenvironment{mykeywords}{\vspace{1em}\begin{adjustwidth}{3em}{3em}\begin{small}\textbf{Keywords.}}{\end{small}\end{adjustwidth}\vspace{1em}}

\DeclareCaptionLabelSeparator{custom}{.}
\DeclareCaptionLabelFormat{custom}
{%
  \textsc{#1 #2}
}
\DeclareCaptionFormat{custom}
{%
  #1#2 #3
}
\captionsetup
{
  format=custom,%
  labelformat=custom,%
  labelsep=custom
}

\begin{document}

\begin{center}
  {\Large\bfseries Orthogoval Notes}
\end{center}

\subsection*{Intersection Enumeration}

Let $\mathbf{D}_i=(V,B_i)$, $j=1,\,2,\dots,\,\alpha$, be Desarguesian affine
planes of order $2^n$ on the same point set where the blocks of $\mathbf{D}_i$
are ovals in $\mathbf{D}_j$ for $i \neq j$; and let $\mathbf{D} = \mathbf{D}_1
\cup \mathbf{D}_2 \cup \cdots \cup \mathbf{D}_\alpha$. Then $\mathbf{D}$ is a
resolvable 2--design with parameters
\begin{align*}
  v &= 2^n, & b &= \alpha 2^n(2^n+1), & r &= \alpha(2^n+1), & k &= 2^n, & \lambda &= \alpha.
\end{align*}

Label the parallel classes of $\mathbf{D}$ as
$\mathscr{C}_1,\,\mathscr{C}_2,\dots,\,\mathscr{C}_r$. Fix a block $B$. Then
\[
  \abs{B \cap B'} =
  \begin{cases}
    0 & \text{if $B,\,B' \in \mathscr{C}_i$; and} \\
    \text{0, 1, or 2} & \text{$B \in \mathscr{C}_i$, $B' \in \mathscr{C}_j$, $i
                        \neq j$.}
  \end{cases}
\]
Let $n_0,\,n_1,\,n_2$ be the number of blocks distinct from $B$ which intersect
it in 0, 1, or 2 points. The usual variance counting yields the nonsingular
triangular system
\begin{align*}
  n_0+n_1+n_2 &= \alpha 2^n(2^n+1) - 1, \\
  n_1+2n_2 &= 2^n\Big( \alpha(2^n+1) - 1 \Big), \\
  2n_2 &= 2^n(2^n-1)(\alpha-1).
\end{align*}
The unique solution is given by
\begin{align*}
  n_0 &= 2^{n-1}\Big( (2^n-1)(\alpha-1)+2 \Big)-1, \\
  n_1 &= 2^{n+1}(2^{n-1}+\alpha-1), \\
  n_2 &= 2^{n-1}(2^n-1)(\alpha-1).
\end{align*}

\noindent The number of blocks not in the $\mathscr{C}_i$ containing $B$ that
are disjoint to $B$ is then
\[
  n_0-2^n+1 = n_2.
\]

\noindent The number of blocks that intersect $B$ in a single point outside of
the $\mathbf{D}_j$ containing $B$ is
\[
  n_1 - 2^{2n} = 2^{n+1}(\alpha-1).
\]

\end{document}
